%! TEX program = lualatex
\documentclass[10pt]{article}

\usepackage{geometry}
\usepackage{enumitem}
\usepackage{fontspec}
\usepackage{xifthen}
\usepackage{multirow}
\usepackage{xcolor}
\usepackage[none]{hyphenat}
\usepackage{fontawesome}

\usepackage[hidelinks]{hyperref}
\urlstyle{same} % style hyperlinks like regular text

\pagestyle{empty}
\geometry{margin=0.5in}
\setlength\parindent{0pt}
\setlist{nosep}
\setmainfont{Roboto-Light}
\setlength{\arrayrulewidth}{1.5pt}
\setlength{\tabcolsep}{10pt}

\newfontfamily\roboto{Roboto-Regular}
\newfontfamily\robotothin{Roboto-Thin}
\newcommand{\lmr}[1]{{\fontfamily{lmr}\selectfont#1}}

\newcommand{\name}[2]{
  \begin{minipage}{0.4\textwidth}
    \begin{flushright}
      \setlength{\tabcolsep}{0pt}
      \begin{tabular}{r}
        \fontsize{30pt}{0pt}\selectfont
        \MakeUppercase{#1 \textbf{#2}} \\
        \normalsize \, \textit{Software Engineer}\hfill % \url{https://dav.sh/} \,
      \end{tabular}
    \end{flushright}
  \end{minipage}
}

\newcommand{\textlbf}[1]{{\roboto #1}}
\newcommand{\textlight}[1]{{\robotothin #1}}
\renewcommand{\date}[2]{#1 #2}
\newcommand{\daterange}[2]{#1 -- \ifthenelse{\equal{#2}{}}{\textit{Present}}{#2}}
\newcommand{\resumesection}[1]{\vspace{-0.3cm}\section*{#1}\vspace{-0.3cm}\vspace{0.1cm}}
\newcommand{\heading}[2]{\textbf{#1} \\ \textit{#2}}

\begin{document}

{
\setlength{\tabcolsep}{20pt}
\begin{center}
  \begin{tabular}{c | c}
% \name is self-explanatory. \topinfo is for things like social media, email, phone number, address, etc.
  \name{David}{Chen} & \begin{minipage}{0.4\textwidth}
    \vspace{6pt}
    \href{mailto:chendavid2012@gmail.com}{chendavid2012@gmail.com}  \\
    \faGithub \, \href{https://github.com/nehc9050}{/nehc9050} \\
    \faLinkedin \, \href{https://www.linkedin.com/in/dcychen/}{/in/dcychen} \\
    (781) 367-3554
    \vspace{6pt}
  \end{minipage}
\end{tabular}
\end{center}
}

\vspace{0.5cm}

\begin{minipage}[t]{0.25\textwidth}
  \begin{flushleft}
    \resumesection{Education}
    \textbf{Harvard College} \newline
    \textit{B.A. in Computer Science} \newline
    \textit{Secondary in Music} \newline
    \textlbf{GPA:} 3.83 \newline
    \daterange{\date{Sept}{2018}}{\date{May}{2022}} \textit{\color{gray}{(exp.)}}

    \vspace{0.25cm}
    
    \resumesection{Skills}
    \textbf{Languages} \newline
    Python \newline
    JavaScript \newline
    C and C++ \newline
    Java \newline
    HTML + CSS \newline
    OCaml

    \vspace{0.25cm}

    \textbf{Technologies} \newline
    Git \newline
    Node.js \newline
    React + Redux \newline
    Deck.gl \newline
    D3 \newline
    Docker \newline
    Django
    
    \vspace{0.25cm}

    \resumesection{Awards}
    \textbf{Codestellation} \newline
    Judge's Pick \newline
    \date{Nov 10}{2019}

    \vspace{0.15cm}

    Best Web/Mobile Hack \newline
    \date{Nov 4}{2018}
    
    \vspace{0.25cm}
    
    \resumesection{Coursework}
    Intro to CS \newline
    Abstraction and Design in \\ \hspace{7} Computation \newline
    Multivariable Calculus \newline
    Statistics and Probability \newline
    Linear Algebra \newline
    Linear Algebra and Big Data \newline
    Discrete math \newline
    Computational Theory \newline
    Algorithms \& Data Structures \newline

  \end{flushleft}
\end{minipage}
\hfill
\begin{minipage}[t]{0.7\textwidth}
  \begin{flushleft}

  \resumesection{Experience}
  \textbf{Grove Collaborative} \, DXM \hfill San Francisco, CA \, \textbf{Remote} \newline
  \textit{Systems Engineering Intern} \hfill \daterange{\date{Jun}{2020}}{\date{Aug}{2020}}
  \begin{itemize}
    \item Improved internal Docker-based web development platform (DXM) by integrating multiple services, including ElasticSearch and Memcached, automating multiple setup processes, adding debugging support, and revamping old code
    \item Created a local developer monitoring service using Prometheus from the ground up for company’s main website
  \end{itemize}

  \vspace{0.25cm}

  \textbf{Rocket Software} \, Zowe App Platform \hfill \textbf{Waltham, MA} \newline
  \textit{Software Engineering Intern} \hfill \daterange{\date{Jun}{2019}}{\date{Aug}{2019}}
  \begin{itemize}
    \item Worked with React.js GraphQL, and i18N to enable various functionalities and improve application extensibility for a commercial web application for mainframe modernization and db2 querying
    \item Built a new web interface for mainframe visualization with filter, search, and manual collection creation capabilities for db2 database objects.
  \end{itemize}

  \vspace{0.25cm}

  \textbf{Federal Reserve Bank of Boston} \hfill \textbf{Boston, MA} \newline
  \textit{Software Engineering Intern} \hfill \daterange{\date{Jul}{2018}}{\date{Aug}{2018}}
  \begin{itemize}
    \item Utilized Python + Tensorflow in a team of two to develop a facial recognition app
    \item Created algorithm to collect data (pictures) for training input and enabled real-time facial detection capabilities using OpenCV; end product deployed for use with bank security
  \end{itemize}

  \vspace{0.25cm}

  \textbf{Federal Reserve Bank of Boston} \hfill \textbf{Boston, MA} \newline
  \textit{Software Engineering Intern} \hfill \daterange{\date{Jul}{2017}}{\date{Aug}{2017}}
  \begin{itemize}
    \item Used Amazon Web Services to develop an Alexa app to display public bank data and provide analytics by parsing interal JSON inputs based off of custom queries created from user input; end product presented to bank executives
  \end{itemize}

  \resumesection{Extracurriculars}
  \textbf{CS50} \, Introduction to Computer Science Course \hfill \textbf{Cambridge, MA} \newline
  \textit{Course Assistant} \hfill \daterange{\date{Sept}{2019}}{\date{Dec}{2019}}
  \begin{itemize}
    \item Hold weekly sessions teaching students programming (including C, Python, HTML/CSS, Javascript, SQL) and core computer science concepts
  \end{itemize}
  
  \vspace{0.25cm}
  \textbf{Harvard Crimson} \, School Newspaper \hfill \textbf{Cambridge, MA} \newline
  \textit{Tech Associate} \hfill \daterange{\date{May}{2019}}{}
  \begin{itemize}
    \item Aided in porting old codebase to new React.js framework by implementing shortcodes and creating custom article and page templates
    \item Manage django backend and GraphQL to work with new React.js framework
  \end{itemize}
  
  
  \resumesection{Projects}

  \textbf{WikiWhere} \hfill \href{https://wikiwhere.org/}{\faLink \, \textit{https://wikiwhere.org/}} \quad \href{https://github.com/wikiwhere/wikiwhere}{\faGithub \, \textit{wikiwhere/wikiwhere}} \newline
  A graph-based visualization for Wikipedia articles (edges represent links). Features include finding shortest paths between two articles using bi-directional BFS that queries Wikipedia data dumps and traversing through linked articles. Technologies used include d3 for graph visualization, and JavaScript and C++ for the backend. Primarily worked with d3, getting everything to show up in the frontend.

  \vspace{0.25cm}

  \textbf{hackm.app} \hfill \href{https://hackm.app/}{\faLink \, \textit{https://hackm.app/}} \quad \href{https://github.com/hackmapp/hackmapp}{\faGithub \, \textit{hackmapp/hackmapp}} \newline
  An interactive map of the previous, current, and upcoming MLH hackathons. Technologies include Python and BeautifulSoup for web scraping, Express for the backend, and d3 for visualization. Worked with d3 in the frontend.

  \end{flushleft}
\end{minipage}

\end{document}
